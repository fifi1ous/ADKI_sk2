\section{Problematické situace}

\subsection{Transformace souřadnic bodů ze souboru}

Při načítání souřadnicových dat ze souborů (např. \texttt{.txt} nebo \texttt{.xyz}) do aplikace je nutné body transformovat tak, aby odpovídaly souřadnicovému systému vykreslovacího okna. Data obvykle pocházejí z reálného terénu a mají velké absolutní hodnoty souřadnic. Tyto hodnoty je třeba převést (škálovat) na oblast viditelnou v aplikaci.

\begin{itemize}
    \item \textbf{Načtení bodů ze souboru} \\
    Načítají se trojice souřadnic $(x, y, z)$, přičemž aplikace podporuje formát oddělený čárkou, mezerou nebo jejich kombinací. Před načtením se vymažou předchozí body.

    \item \textbf{Zjištění rozsahu hodnot (bounding box)} \\
    Během načítání jsou pro každý bod určeny minimální a maximální hodnoty souřadnic $x$, $y$ a $z$, které později slouží k výpočtu měřítka.

    \item \textbf{Zjištění rozměrů vykreslovacího okna} \\
    Pomocí funkcí \texttt{width()} a \texttt{height()} jsou získány rozměry okna aplikace, do kterého se mají data zobrazit.

    \item \textbf{Výpočet měřítka} \\
    Souřadnice bodů se transformují pomocí škálování. Měřítko se počítá zvlášť pro osu $x$ a $y$:
    \begin{equation}
        \text{scaleX} = \frac{\text{windowWidth}}{\text{maxX} - \text{minX}}, \quad
        \text{scaleY} = \frac{\text{windowHeight}}{\text{maxY} - \text{minY}}
    \end{equation}

    \item \textbf{Výpočet posunu (offsetu)} \\
    Aby byly body zarovnány k okraji okna, použije se posun:
    \begin{equation}
        \text{offsetX} = -\text{minX}, \quad \text{offsetY} = -\text{minY}
    \end{equation}

    \item \textbf{Transformace souřadnic bodů} \\
    Každému bodu je následně vypočítána nová souřadnice takto:
    \begin{equation}
        \text{transformedX} = (x + \text{offsetX}) \cdot \text{scaleX}
    \end{equation}
    \begin{equation}
        \text{transformedY} = (y + \text{offsetY}) \cdot \text{scaleY}
    \end{equation}

    \item \textbf{Uložení bodů a překreslení} \\
    Po transformaci jsou nové souřadnice uloženy do vektoru a aplikace je překreslena funkcí \texttt{repaint()}.
\end{itemize}
