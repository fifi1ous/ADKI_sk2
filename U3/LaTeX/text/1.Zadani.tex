\section{Zadání}

Zadáním této úlohy bylo metodou inkrementální konstrukce vytvořit digitální model terénu (DMT) nad množinou 3D bodů pomocí 2D Delaunay triangulace. Výstupem je polyedrický DMT reprezentovaný vrstevnicemi, doplněný o vizualizaci sklonu trojúhelníků a jejich expozice.

\begin{itemize}
    \item \textbf{Vstup:} Množina bodů \( P = \{ p_1, p_2, \dots, p_n \} \), kde každý bod \( p_i \) je reprezentován třemi souřadnicemi \( p_i = (x_i, y_i, z_i) \), kde \( x_i \) a \( y_i \) jsou prostorové souřadnice bodu v rovině a \( z_i \) je výška nad referenčním bodem.
    \item \textbf{Výstup:} Polyedrický digitální model terénu (DMT), reprezentovaný triangulacemi a vrstevnicemi, spolu s vizualizacemi sklonu a expozice terénu.
\end{itemize}

\begin{itemize}
    \item Delaunay triangulace, polyedrický model terénu.
    \item Konstrukce vrstevnic, analýza sklonu a expozice.
\end{itemize}