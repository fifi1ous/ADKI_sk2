\section{Závěr}
V rámci této úlohy byla vytvořena aplikace, která umožňuje analýzu terénu. Aplikace dokáže analyzovat data ručně zadaná do aplikace, automaticky generovaná aplikací, nebo načítáním dat z formátu \texttt{.txt} nebo \texttt{.xyz}.Výsledky jsou reprezentovány ve 2D pomocí TIN, vrstevnic, sklonu a orientace terénu. Body terénu mohou být zobrazeny ve 3D prohlížeči.
Vzhledem k časovému vytížení v rámci jiných předmětů, zejména projektu z kartografie, jsme nestihli zpracovat všechny bonusové úlohy, které bychom si jinak rádi splnili.

\subsection{Omezení Delaunay triangulace}

Přestože je Delaunay triangulace velmi užitečná pro tvorbu digitálního modelu terénu, v určitých případech není její použití ideální. Mezi situace, kdy nemusí poskytovat dobré výsledky, patří:
\begin{itemize}
\item \textbf{Zlomové linie a strmé hrany:} Delaunay triangulace nedokáže sama o sobě respektovat morfologické zlomy, jako jsou prudké srázy. Bez použití povinných hran dochází k chybné interpretaci.
\item \textbf{Rovné plochy:} V oblastech s velmi malým nebo žádným sklonem nemusí vznikat žádné vrstevnice, nebo naopak mohou vznikat chybné vrstevnice.
\item \textbf{Špatné rozložení:} Pokud jsou vstupní data špatně rozložená a vznikají dlouhé a úzké trojúhelníky, může to vést k nepřesné interpolaci a neplynulým vrstevnicím.
\item \textbf{Pravidelná mřížka:} V případě pravidelně rozmístěných bodů může vzniknout nejednoznačná triangulace, což může způsobit chyby ve tvaru vrstevnic.
\end{itemize}
Pro tyto případy by bylo vhodné rozšířit aplikaci o podporu povinných hran, které umožňují zachování tvaru morfologicky významných útvarů. Dále by bylo možné přidat další metody interpolace, které v určitých případech poskytují lepší výsledky než Delaunay triangulace.

\subsection{Další Možné Neřešené Problémy a Náměty na Vylepšení}

\begin{itemize} 
    \item \textbf{Podpora dalších formátů pro vstup:} V současné verzi aplikace je možné načítat bodová data pouze z formátů TXT a XYZ, přičemž jako oddělovač slouží čárka nebo mezera. Do budoucna by bylo vhodné rozšířit možnosti importu o podporu volitelného nastavení formátu (včetně výběru oddělovače, hlaviček apod.) a přidat podporu běžně používaných formátů pro práci s mračny bodů, jako jsou LAS a LAZ.
    
    \item \textbf{Možnost exportu dat:} Aplikace aktuálně neumožňuje export výsledků. Do budoucna by bylo vhodné implementovat podporu pro ukládání výsledků do běžně používaných GIS formátů, jako je SHP nebo GeoPackage, a zároveň umožnit export vizualizací do rastrových (např. TIFF) nebo vektorových (např. SVG) formátů.
    
    \item \textbf{Dávkové zpracování:} Dalším vylepšením by mohlo být aplikaci kromě zpracování v GUI umožnit i dávkové zpracování mračen bodů v příkazové řádce, které by mohlo vypadat například takto:\\ \texttt{DMT.exe -TIN mracno.txt > tin.shp}\\ \texttt{DMT.exe -contourLines mracno.xyz > vrstevnice.shp}.
    
    \item \textbf{Zoomování a posouvání mapy:} Aplikace v současnosti nepodporuje přibližování a pohyb v mapovém okně.

    \item \textbf{Povinné hrany} Možnost nastavení povinných hran.

    \item \textbf{Nastavení:} Přidat další možnosti nastavená, kde se budou moc nastavit vlastní parametry barevného zobrazení sklonu a orientace terénu.

    \item \textbf{Vizuál aplikace:} Zlepšit vizuální stránku aplikace.

\end{itemize}
