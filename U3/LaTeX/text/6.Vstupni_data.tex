\section{Vstupní data}

Vstupní data pro tvorbu digitálního modelu terénu mohou být do aplikace zadána několika způsoby, což umožňuje flexibilní testování i realistické modelování různých terénních situací:

\begin{itemize}
    \item \textbf{Ručně uživatelem} – body lze vkládat interaktivně kliknutím do grafického rozhraní. Souřadnice $x$ a $y$ odpovídají poloze kurzoru, výška $z$ je generována náhodně.

    \item \textbf{Generování syntetických dat} – aplikace umožňuje automaticky vygenerovat reprezentativní morfologické tvary terénu:
    \begin{itemize}
        \item \emph{Kupa}
        \item \emph{Údolí}
        \item \emph{Hřbet}
        \item \emph{Spočinek}
        \item \emph{Sedlo}
    \end{itemize}

    \item \textbf{Načtení ze souboru} – aplikace podporuje import bodových dat ze souborů ve formátu \texttt{.txt} nebo \texttt{.xyz}. Každý řádek obsahuje trojici hodnot: $x$, $y$, $z$, přičemž hodnoty mohou být odděleny mezerou nebo čárkou. Jako oddělovač desetinných míst se používá tečka. Ukázka vstupního souboru:

    \begin{verbatim}
120.0,280.0,320.0
160.0,670.0,370.0
260.0,310.0,410.0
...
    \end{verbatim}

Vzorková data použitá v této úloze pocházejí z digitálního modelu reliéfu České republiky – DMR 5G. Zdroj: \textbf{Český úřad zeměměřický a katastrální (ČÚZK)} – dostupné online na: Zdroj dat: \href{https://geoportal.cuzk.cz/}{Český úřad zeměměřický a katastrální (ČÚZK)}~\cite{cuzk}.

\end{itemize}