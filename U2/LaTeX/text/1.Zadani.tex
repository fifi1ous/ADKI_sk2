\section{Zadání}

Zadáním úlohy bylo implementovat generalizaci budov do úrovně detailu LOD0.

\begin{itemize}
\item Vstup: množina budov \( B = \{B_i\}_{i=1}^n \), kde budova \( B_i \) je reprezentována množinou lomových bodů \( \{P_{i,j}\}_{j=1}^m \).
\item Výstup: \( G(B_i) \).

\item Ze souboru načtěte vstupní data představovaná lomovými body budov a proveďte generalizaci budov do úrovně detailu LOD0. Pro tyto účely použijte vhodnou datovou sadu, například ZABAGED. Testování proveďte nad třemi datovými sadami (historické centrum města, sídliště, izolovaná zástavba).

\item Pro každou budovu určete její hlavní směry metodami:
    \begin{itemize}
        \item Minimum Area Enclosing Rectangle,
        \item PCA.
    \end{itemize}

\item U první metody použijte některý z algoritmů pro konstrukci konvexní obálky. Budovu při generalizaci do úrovně LOD0 nahraďte obdélníkem orientovaným v obou hlavních směrech, se středem v těžišti budovy, jehož plocha bude stejná jako plocha budovy. Výsledky generalizace vhodně vizualizujte.

\item Otestujte a porovnejte efektivitu obou metod s využitím hodnotících kritérií. Pokuste se rozhodnout, pro které tvary budov dávají metody nevhodné výsledky, a pro které naopak poskytují vhodnou aproximaci.
\end{itemize}
