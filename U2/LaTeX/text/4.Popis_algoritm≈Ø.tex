\section{Popis algoritmů}

\subsection{Algoritmy pro tvorbu konvexní obálky}

\subsubsection{Jarvis scan}
Jarvisův sken, známý také jako Gift Wrapping Algorithm, je algoritmus sloužící ke konstrukci konvexní obálky z množiny bodů v rovině. Princip spočívá v postupném „obalování“ bodů – algoritmus vždy přidává takový bod, který tvoří s aktuálním směrem největší úhel.\\

\hspace{-1.15cm}
Nejprve se začne vybráním pivotu $q$, bodu s nejmenší $y$ souřadnicí:
\begin{equation}
    q = \text{min}(y_i)
\end{equation}
Následně se vybere bod $r$ s nejmenší $x$ souřadnicí:
\begin{equation}
    r = \text{min}(x_i)
\end{equation}
Jou inicializovány body $p_j$ a $p_{j1}$, $p_{j1}$ tvoří posunutá $x$ souřadnice bodu $r$ a $y$ souřadnice bodu $y$:
\begin{equation}
    p_j = q
\end{equation}
\begin{equation}
    p_{j1}(r.x-1,q.y)
\end{equation}
Pivot $q$ je přidán do do konvexní obálky $ch$:
\begin{equation}
    ch.\text{push\_back}(q)
\end{equation}
Následně se projdou všechny body a najde se takový, který tvoří svírá s $p{j}$ a $p_{j-1}$ maximální úhel. Výběr je provázen pouze u bodů, které již nejsou v $ch$:
\begin{equation}
    p_{j+1} = \text{max} \angle( p_{j-1},p_j, p_i)
\end{equation}
Bod je následně do konvexní obálky:
\begin{equation}
    ch.\text{push\_back}(p_{j+1})
\end{equation}

\begin{algorithm}
    \setstretch{0.9}
    \caption{Metoda \texttt{Jarvis Scan}}
    \begin{algorithmic}[1]
        \STATE \textbf{Vstup:} polygon $P$
        \STATE \textbf{Výstup:} konvexní obal polygonu $P$

        \STATE $\text{CH} \gets \emptyset$
        \IF{$\text{size}(P) = 3$}
            \STATE \textbf{return} $P$
        \ENDIF

        \STATE $q \gets \min(P)$ podle $y$-ové souřadnice
        \STATE $r \gets \min(P)$ podle $x$-ové souřadnice

        \STATE $p_j \gets q$
        \STATE $p_{j1} \gets (r_x - 1, q_y)$
        \STATE \text{přidej } $p_j$ do $\text{CH}$

        \REPEAT
            \STATE $\omega_{\text{max}} \gets 0$
            \STATE $i_{\text{max}} \gets -1$

            \FOR{$i \gets 0$ \TO $\text{size}(P)-1$}
                \STATE $\omega \gets \text{get2LinesAngle}(p_j, p_{j1}, p_j, P[i])$
                \IF{$\omega > \omega_{\text{max}}$}
                    \STATE $\omega_{\text{max}} \gets \omega$
                    \STATE $i_{\text{max}} \gets i$
                \ENDIF
            \ENDFOR

            \STATE přidej $P[i_{\text{max}}]$ do $\text{CH}$
            \STATE $p_{j1} \gets p_j$
            \STATE $p_j \gets P[i_{\text{max}}]$
        \UNTIL{$p_j = q$}

        \STATE \textbf{return} $\text{CH}$
    \end{algorithmic}
\end{algorithm}

\subsubsection{Graham scan}
Grahamův scan je algoritmus, který vytváří konvexní obálku množiny bodů. Algoritmus začíná dvěma počátečními body $p_1$ a $p_2$. Pro každý kandidátský bod $k$ se zjišťuje, zda leží vlevo od přímky tvořené body $p_1$ a $p_2$. Pokud ano, bod $k$ se přidá do konvexní obálky. Pokud však leží vpravo od této přímky, bod $p_2$ je z obálky odebrán a nahrazen kandidátem $k$.\\

\hspace{-1.15cm}
Nejprve je nalezen bod $q$ který má minimální $y$ souřadnici: 
\begin{equation}
    q = \text{min}(y_i)
\end{equation}
Pokud je více bodů, které mají stejnou $y$ souřadnici, vybere se bod, který má menší $x$ souřadnici:
\begin{equation}
    \text{if}~\text{pocet}~q>1~\text{then}~q = {min}(x_q)
\end{equation}
Následně je vypočítán od přímky která tvoří je rovnoběžná s osou $x$ a prochází bodem $q$ úhel ke všem ostatním bodům v množině:
\begin{equation}
    \omega _i = \angle( q.x+1,q,p_i)
\end{equation}
Všechny body $p_i$ jsou seřazeny sestupně podle úhlů~$\omega_i$ vzhledem k pivotu~$q$, kde platí:
\begin{equation}
    p = \text{sort}(\omega,\downarrow)
\end{equation}
Pokud mají dva body stejný úhel $\omega_i = \omega_j$, je ponechán pouze ten, který má větší vzdálenost od pivota~$q$, tedy ten pro nějž platí.\\
Následně jsou do konvexní obálky přidány body $q$,$p_1$,$p_2$:
\begin{equation}
    ch.\text{push\_back}(q,p_1,p_2)
\end{equation}
Projdou se všechny zbylé body a pokud $p_{i}$ je vlevo od přímky tvořenou $p_{i-1}$ a $p_{i-2}$. Je přidán do $ch$:
\begin{equation}
    ch.\text{push\_back}(p_{i})
\end{equation}
Pokud je pokud bod $p_{i}$ vpravo od přímky tvořenou $p_{i-1}$ a $p_{i-2}$. Je odebrán bod $p_{i-1}$ a přidán $p_{i}$:
\begin{equation}
    ch.\text{pop\_back}(p_{i-1})
\end{equation}
\begin{equation}
    ch.\text{push\_back}(p_{i})
\end{equation}
\begin{algorithm}
    \setstretch{0.8}
    \caption{Metoda \texttt{createCHGS}}
    \begin{algorithmic}[1]
        \STATE \textbf{Vstup:} polygon $P$
        \STATE \textbf{Výstup:} konvexní obal polygonu $P$

        \STATE $\text{CH} \gets \emptyset$
        \IF{$\text{size}(P) = 3$}
            \STATE \textbf{return} $P$
        \ENDIF

        \STATE $q \gets \text{findPivotGS}(P)$ \COMMENT{nejnižší bod podle $y$}

        \STATE $P' \gets P \setminus \{q\}$ \COMMENT{odstraň pivot z množiny bodů}

        \STATE $\alpha \gets \text{anglesWithPoints}(P', q)$ \COMMENT{úhly vůči pivotu}
        \STATE \text{sortAnglesPoints}$(q, \alpha, P')$ \COMMENT{seřazení bodů podle úhlu}

        \STATE přidej $q$, $P'[0]$, $P'[1]$ do $\text{CH}$

        \FOR{$i \gets 2$ \TO $\text{size}(P') - 1$}
            \STATE $c \gets P'[i]$
            \WHILE{$\text{size}(CH) \geq 2$ \AND $\text{findSide}(CH_{-2}, CH_{-1}, c) = -1$}
                \STATE \text{odstraň poslední bod z } $\text{CH}$
            \ENDWHILE
            \STATE přidej $c$ do $\text{CH}$
        \ENDFOR

        \STATE \textbf{return} $\text{CH}$
    \end{algorithmic}
\end{algorithm}

\begin{algorithm}
    \setstretch{0.8}
    \caption{Metoda \texttt{findPivotGS} -- hledání pivotu pro Graham Scan}
    \begin{algorithmic}[1]
        \STATE \textbf{Vstup:} polygon $P$
        \STATE \textbf{Výstup:} pivotní bod $q$ s nejmenší $y$ souřadnicí (případně nejmenší $x$)

        \STATE $q \gets P[0]$
        \STATE $q_y \gets y$-ová souřadnice bodu $q$

        \FOR{$i \gets 1$ \TO $\text{size}(P) - 1$}
            \STATE $c \gets P[i]$
            \STATE $c_y \gets y$-ová souřadnice bodu $c$

            \IF{$c_y \leq q_y$}
                \IF{$c_y = q_y$}
                    \STATE $c_x \gets x$-ová souřadnice bodu $c$
                    \STATE $q_x \gets x$-ová souřadnice bodu $q$
                    \IF{$c_x < q_x$}
                        \STATE $q \gets c$
                        \STATE $q_y \gets c_y$
                    \ENDIF
                \ELSE
                    \STATE $q \gets c$
                    \STATE $q_y \gets c_y$
                \ENDIF
            \ENDIF
        \ENDFOR

        \STATE \textbf{return} $q$
    \end{algorithmic}
\end{algorithm}

\begin{algorithm}
    \setstretch{0.8}
    \caption{Metoda \texttt{sortAnglesPoints} -- třídění bodů podle úhlů od pivotu}
    \begin{algorithmic}[1]
        \STATE \textbf{Vstup:} pivotní bod $q$, vektor úhlů $\alpha$, polygon $P'$
        \STATE \textbf{Výstup:} seřazené a vyfiltrované $\alpha$ a $P'$

        \STATE vytvoř vektor indexů $I \gets [0, 1, \dots, n-1]$

        \STATE vytvoř vektor indexů $\alpha_{\text{sorted}}$
        \STATE $P'_{\text{sorted}} \gets []$

        \IF{$I$ není prázdný}
            \STATE přidej $\alpha[I_0]$ do $\alpha_{\text{sorted}}$
            \STATE přidej $P'[I_0]$ do $P'_{\text{sorted}}$

            \FOR{$i \gets 1$ \TO $\text{size}(I) - 1$}
                \STATE $i_{\text{cur}} \gets I[i]$
                \STATE $i_{\text{prev}} \gets I[i - 1]$

                \IF{$\alpha[i_{\text{cur}}] \neq \alpha[i_{\text{prev}}]$}
                    \STATE přidej $\alpha[i_{\text{cur}}]$ do $\alpha_{\text{sorted}}$
                    \STATE přidej $P'[i_{\text{cur}}]$ do $P'_{\text{sorted}}$
                \ELSE
                    \STATE $d_{\text{cur}} \gets \text{vzdálenost}(P'[i_{\text{cur}}], q)$
                    \STATE $d_{\text{prev}} \gets \text{vzdálenost}(P'_{\text{sorted}}.\text{last}, q)$
                    \IF{$d_{\text{cur}} > d_{\text{prev}}$}
                        \STATE nahraď poslední bod v $P'_{\text{sorted}}$ bodem $P'[i_{\text{cur}}]$
                    \ENDIF
                \ENDIF
            \ENDFOR
        \ENDIF

        \STATE $\alpha \gets \alpha_{\text{sorted}}$
        \STATE $P' \gets P'_{\text{sorted}}$
    \end{algorithmic}
\end{algorithm}


\subsection{Algoritmy pro generalizaci budov}

\subsubsection{Minimum Area Enclosing Rectangle}
Tento algoritmus slouží k nalezení obdélníku s nejmenší plochou, který obsahuje všechny body daného polygonu. Využívá se přitom konstrukce konvexní obálky, která se postupně otáčí podle orientace jednotlivých jejích hran. Pro každou rotaci se vypočítá min-max box a vybírá se ten s nejmenší plochou. Po nalezení min-max boxu s nejmenší plochou se tento min-max box rotuje zpět do původní orientace. Výsledný obdélník je dále upraven tak, aby měl stejný obsah jako původní polygon a zachoval jeho těžiště\cite{Bayer2008}.\\

\hspace{-1.15cm}
Nejprve se vytvoří konvexní obálka \( CH \) z množiny bodů budovy \( P \):
\begin{equation}
    CH = \text{createCH}(P)
\end{equation}
Pro každou hranu konvexní obálky mezi body \( p_i \) a \( p_{i+1} \) se vypočítá úhel rotace \( \sigma \):
\begin{equation}
    \sigma = \arctan2(y_{i+1} - y_i, x_{i+1} - x_i)
\end{equation}
Konvexní obálka se otočí o úhel \( -\sigma \):
\begin{equation}
    CH_{\text{rot}} = \mathbf{R}(-\sigma) CH
\end{equation}
Kde \textbf{R}(-\(\sigma\)) je rotační matice:
\[
\mathbf{R}(\-\sigma) = 
\begin{bmatrix}
\cos\-\sigma & -\sin\-\sigma \\
\sin\-\sigma & \cos\-\sigma
\end{bmatrix}
\]
Po rotaci se vypočítá min-max box a jeho plocha:
\begin{align}
    \text{MMBox} &= \left\{
    \begin{bmatrix} x_{\min} \\ y_{\min} \end{bmatrix},
    \begin{bmatrix} x_{\max} \\ y_{\min} \end{bmatrix},
    \begin{bmatrix} x_{\max} \\ y_{\max} \end{bmatrix},
    \begin{bmatrix} x_{\min} \\ y_{\max} \end{bmatrix}
    \right\}\\
    \nonumber\\
    \text{area} &= (x_{\max} - x_{\min})(y_{\max} - y_{\min})
\end{align}
Pokud je plocha menší než aktuálně nejmenší nalezená:
\begin{equation}
    \text{if } \text{area} < \text{area}_{\min} \text{ then } 
    \text{area}_{\min} = \text{area},\quad
    \sigma_{\min} = \sigma,\quad
    \text{MMBox}_{\min} = \text{MMBox}
\end{equation}
Po projití všech hran se spočítá plocha původního polygonu pomocí L'Huilierových vzorců:
\begin{equation}
    A_P = \frac{1}{2} \left| \sum_{i=1}^{n} x_i(y_{i+1} - y_{i-1}) \right|
\end{equation}
Min-max box se zmenší tak, aby měl stejný obsah jako původní polygon budovy\( A_P \). Nejprve se spočítá těžiště:
\[
\mathbf{c} = \frac{1}{4} \sum_{j=1}^4 \mathbf{v}_j, \quad \mathbf{v}_j \in \text{MMBox}_{\min}
\]
Pak se každý bod redukuje k těžišti:
\begin{equation}
    \mathbf{q}_j = \mathbf{c} + \sqrt{\frac{A_P}{\text{area}_{\min}}} \cdot (\mathbf{v}_j - \mathbf{c})
\end{equation}
Výsledný zmenšený box:
\[
\text{MMBox}_{\text{res}} = \{ \mathbf{q}_1, \mathbf{q}_2, \mathbf{q}_3, \mathbf{q}_4 \}
\]
Nakonec se box otočí zpět o úhel \( \sigma_{\min} \):
\begin{equation}
    \text{MAER} = \left\{ \mathbf{R}(\sigma_{\min}) \cdot \mathbf{q}_j \mid j = 1,\dots,4 \right\}
\end{equation}

\begin{algorithm}
    \setstretch{0.9}
    \caption{Metoda \texttt{Minimum Area Enclosing Rectangle}}
    \begin{algorithmic}[1]
        \STATE \textbf{Vstup:} polygon $P$
        \STATE \textbf{Výstup:} minimální obdélník obsahující všechny body polygonu $P$
        
        \STATE $\sigma_{\text{min}} \gets 2\pi$
        \STATE $\text{mmbox\_min}, \text{area\_min} \gets \text{minMaxBox}(P)$
        \STATE $CH \gets \text{createCH}(P)$
        \STATE $n \gets \text{size}(CH)$
        
        \FOR{$i \gets 0$ \TO $n-1$}
            \STATE $dx \gets x_{i+1} - x_i$
            \STATE $dy \gets y_{i+1} - y_i$
            \STATE $\sigma \gets \arctan2(dy, dx)$
            \STATE $CH_{\text{rot}} \gets \text{rotate}(CH, -\sigma)$
            \STATE $\text{mmbox}, \text{area} \gets \text{minMaxBox}(CH_{\text{rot}})$
            
            \IF{$\text{area} < \text{area\_min}$}
                \STATE $\text{area\_min} \gets \text{area}$
                \STATE $\sigma_{\text{min}} \gets \sigma$
                \STATE $\text{mmbox\_min} \gets \text{mmbox}$
            \ENDIF
        \ENDFOR
        
        \STATE $\text{mmbox\_min\_res} \gets \text{resize}(P, \text{mmbox\_min})$
        \STATE \textbf{return} \text{rotate}($\text{mmbox\_min\_res}, \sigma_{\text{min}}$)
    \end{algorithmic}
\end{algorithm}

\subsubsection{Principal Component Analysis (PCA)}
Metoda, která určuje hlavní směry budovy pomocí analýzy hlavních komponent. Výsledný obdélník je natočený podle hlavních směrů budovy a má obsah shodný s obsahem budovy. Těžiště obdélníku je shodné s těžištěm polygonu. Pro výpočet hlavních komponent je využita knihovna \href{https://gitlab.com/libeigen/eigen}{eigen}\cite{eigen}.\\

\hspace{-1.15cm}
Nejprve se vytvoří matice \( \mathbf{A} \) z množiny bodů budovy \( P \):
\begin{equation}
    \mathbf{A} = \begin{pmatrix}
    x_1 & y_1 \\
    x_2 & y_2 \\
    \vdots & \vdots \\
    x_n & y_n
    \end{pmatrix}
\end{equation}
Vypočítají se průměry souřadnic přes sloupce:
\begin{equation}
    \mathbf{M} = \frac{1}{n} \sum_{i=1}^{n} \mathbf{A}_i
\end{equation}
Od matice \( \mathbf{A} \) se odečtou průměry \( \mathbf{M} \), čímž vznikne matice \( \mathbf{B} \):
\begin{equation}
    \mathbf{B} = \mathbf{A} - \mathbf{M}
\end{equation}
Vypočítá se kovarianční matice \( \mathbf{C} \):
\begin{equation}
    \mathbf{C} = \frac{\mathbf{B}^T \mathbf{B}}{n - 1}
\end{equation}
Provede se singulární rozklad (SVD) kovarianční matice \( \mathbf{C} \):
\begin{equation}
    \mathbf{C} = \mathbf{U} \mathbf{\Sigma} \mathbf{V}^T
\end{equation}
kde \( \mathbf{U} \) a \( \mathbf{V} \) jsou matice vlastních vektorů a \( \mathbf{\Sigma} \) je diagonální matice singulárních hodnot. \\
\hspace{-1.15cm}
Úhel rotace \( \sigma \) se vypočítá jako:
\begin{equation}
    \sigma = \arctan2(\mathbf{V}_{1,0}, \mathbf{V}_{0,0})
\end{equation}
Polygon \( P \) se otočí o úhel \( -\sigma \):
\begin{equation}
    P_{\text{rot}} = \mathbf{R}(-\sigma) P
\end{equation}
Po rotaci se vypočítá min-max box a jeho plocha:
\begin{align}
    \text{MMBox} &= \left\{
    \begin{bmatrix} x_{\min} \\ y_{\min} \end{bmatrix},
    \begin{bmatrix} x_{\max} \\ y_{\min} \end{bmatrix},
    \begin{bmatrix} x_{\max} \\ y_{\max} \end{bmatrix},
    \begin{bmatrix} x_{\min} \\ y_{\max} \end{bmatrix}
    \right\}\\
    \nonumber\\
    \text{area} &= (x_{\max} - x_{\min})(y_{\max} - y_{\min})
\end{align}
Min-max box se zmenší tak, aby měl stejný obsah jako původní polygon budovy\( A_P \). Nejprve se spočítá těžiště:
\[
\mathbf{c} = \frac{1}{4} \sum_{j=1}^4 \mathbf{v}_j, \quad \mathbf{v}_j \in \text{MMBox}_{\min}
\]
Pak se každý bod redukuje k těžišti:
\begin{equation}
    \mathbf{q}_j = \mathbf{c} + \sqrt{\frac{A_P}{\text{area}_{\min}}} \cdot (\mathbf{v}_j - \mathbf{c})
\end{equation}
Výsledný zmenšený box:
\[
\text{MMBox}_{\text{res}} = \{ \mathbf{q}_1, \mathbf{q}_2, \mathbf{q}_3, \mathbf{q}_4 \}
\]
Nakonec se box otočí zpět o úhel \( \sigma \):
\begin{equation}
    \text{PCA} = \left\{ \mathbf{R}(\sigma) \cdot \mathbf{q}_j \mid j = 1,\dots,4 \right\}
\end{equation}

\begin{algorithm}
    \setstretch{0.9}
    \caption{Metoda \texttt{PCA (Principal Component Analysis)}}
    \begin{algorithmic}[1]
        \STATE \textbf{Vstup:} polygon $P$
        \STATE \textbf{Výstup:} obdélník obsahující všechny body polygonu $P$ pomocí PCA
        
        \STATE $n \gets \text{size}(P)$
        \STATE $\mathbf{A} \gets \text{matrix}(n, 2)$
        
        \FOR{$i \gets 0$ \TO $n-1$}
            \STATE $\mathbf{A}(i,0) \gets P[i].x$
            \STATE $\mathbf{A}(i,1) \gets P[i].y$
        \ENDFOR
        
        \STATE $\mathbf{M} \gets \text{mean}(\mathbf{A})$
        \STATE $\mathbf{B} \gets \mathbf{A} - \mathbf{M}$
        \STATE $\mathbf{C} \gets \frac{\mathbf{B}^T \mathbf{B}}{n - 1}$
        
        \STATE $\mathbf{U}, \mathbf{\Sigma}, \mathbf{V} \gets \text{SVD}(\mathbf{C})$
        
        \STATE $\sigma \gets \arctan2(\mathbf{V}_{1,0}, \mathbf{V}_{0,0})$
        \STATE $P_{\text{rot}} \gets \text{rotate}(P, -\sigma)$
        \STATE $\text{mmbox}, \text{area} \gets \text{minMaxBox}(P_{\text{rot}})$
        
        \STATE $\text{mmbox\_min\_res} \gets \text{resize}(P, \text{mmbox})$
        \STATE \textbf{return} \text{rotate}($\text{mmbox\_min\_res}, \sigma$)
    \end{algorithmic}
\end{algorithm}

\subsubsection{Longest Edge}
Metoda, která určuje natočení obdélníku na základě nejdelší hrany polygonu. Obsah obdélníku je shodný s obsahem polygonu. Těžiště obdélníku je shodné s těžištěm polygonu.\\ Nejprve se najde nejdelší hrana polygonu \( P \):
\begin{equation}
    \text{maxLength} = 0
\end{equation}
Pro každou hranu mezi body \( p_i \) a \( p_{i+1} \) se vypočítají rozdíly souřadnic:
\begin{equation}
    dx_i = x_{i+1} - x_i
\end{equation}
\begin{equation}
    dy_i = y_{i+1} - y_i
\end{equation}
Délka hrany se vypočítá jako:
\begin{equation}
    \text{length} = \sqrt{dx_i^2 + dy_i^2}
\end{equation}
Pokud je délka hrany větší než \( \text{maxLength} \), aktualizuje se \( \text{maxLength} \) a směrové rozdíly \( dx \) a \( dy \):
\begin{equation}
    \text{if } \text{length} > \text{maxLength} \text{ then}
\end{equation}
\begin{equation}
    \quad \text{maxLength} = \text{length}
\end{equation}
\begin{equation}
    \quad dx = dx_i
\end{equation}
\begin{equation}
    \quad dy = dy_i
\end{equation}
Úhel rotace \( \sigma \) se vypočítá jako:
\begin{equation}
    \sigma = \arctan2(dy, dx)
\end{equation}
Polygon \( P \) se otočí o úhel \( -\sigma \):
\begin{equation}
    P_{\text{rot}} = \mathbf{R}(-\sigma) P
\end{equation}
Po rotaci se vypočítá min-max box a jeho plocha:
\begin{align}
    \text{MMBox} &= \left\{
    \begin{bmatrix} x_{\min} \\ y_{\min} \end{bmatrix},
    \begin{bmatrix} x_{\max} \\ y_{\min} \end{bmatrix},
    \begin{bmatrix} x_{\max} \\ y_{\max} \end{bmatrix},
    \begin{bmatrix} x_{\min} \\ y_{\max} \end{bmatrix}
    \right\}\\
    \nonumber\\
    \text{area} &= (x_{\max} - x_{\min})(y_{\max} - y_{\min})
\end{align}
Min-max box se zmenší tak, aby měl stejný obsah jako původní polygon budovy\( A_P \). Nejprve se spočítá těžiště:
\begin{equation}
\mathbf{c} = \frac{1}{4} \sum_{j=1}^4 \mathbf{v}_j, \quad \mathbf{v}_j \in \text{MMBox}_{\min}
\end{equation}
Pak se každý bod redukuje k těžišti:
\begin{equation}
    \mathbf{q}_j = \mathbf{c} + \sqrt{\frac{A_P}{\text{area}_{\min}}} \cdot (\mathbf{v}_j - \mathbf{c})
\end{equation}
Výsledný zmenšený box:
\begin{equation}
\text{MMBox}_{\text{res}} = \{ \mathbf{q}_1, \mathbf{q}_2, \mathbf{q}_3, \mathbf{q}_4 \}
\end{equation}
Nakonec se box otočí zpět o úhel \( \sigma \):
\begin{equation}
    \text{LE} = \left\{ \mathbf{R}(\sigma) \cdot \mathbf{q}_j \mid j = 1,\dots,4 \right\}
\end{equation}

\begin{algorithm}
    \setstretch{0.9}
    \caption{Metoda \texttt{Longest Edge}}
    \begin{algorithmic}[1]
        \STATE \textbf{Vstup:} polygon $P$
        \STATE \textbf{Výstup:} obdélník obsahující všechny body polygonu $P$ pomocí nejdelší hrany
        
        \STATE $\text{maxLength} \gets 0$
        \STATE $dx \gets 0$, $dy \gets 0$
        \STATE $n \gets \text{size}(P)$
        
        \FOR{$i \gets 0$ \TO $n-1$}
            \STATE $dx_i \gets x_{i+1} - x_i$
            \STATE $dy_i \gets y_{i+1} - y_i$
            \STATE $\text{length} \gets \sqrt{dx_i^2 + dy_i^2}$
            
            \IF{$\text{length} > \text{maxLength}$}
                \STATE $\text{maxLength} \gets \text{length}$
                \STATE $dx \gets dx_i$
                \STATE $dy \gets dy_i$
            \ENDIF
        \ENDFOR
        
        \STATE $\sigma \gets \arctan2(dy, dx)$
        \STATE $P_{\text{rot}} \gets \text{rotate}(P, -\sigma)$
        \STATE $\text{mmbox}, \text{area} \gets \text{minMaxBox}(P_{\text{rot}})$
        
        \STATE $\text{mmbox\_min\_res} \gets \text{resize}(P, \text{mmbox})$
        \STATE \textbf{return} \text{rotate}($\text{mmbox\_min\_res}, \sigma$)
    \end{algorithmic}
\end{algorithm}

\subsubsection{Wall Average}
Na každou stranu budovy je aplikována operace mod($\pi/2$). Ze zbytků hodnot je spočten vážený průměr, kde váhou je délka strany. Tato metoda je robustní, avšak citlivá na nepravé úhly. Nejprve jsou určeny směrnice $\sigma_i$ všech hran, poté jsou spočteny vnitřní úhly $\omega_i$.\\
Nejprve se vypočítá počáteční směr $\sigma_{1,2}$:
\begin{equation}
    \sigma_{1,2} = \arctan2(y_2 - y_1, x_2 - x_1)
\end{equation}
Pro každý vrchol $p_i$ se vypočítají směrnice $\sigma_i$ a $\sigma_{i+1}$:
\begin{align}
    \sigma_i &= \arctan2(y_i - y_{i-1}, x_i - x_{i-1}) \\
    \sigma_{i+1} &= \arctan2(y_{i+1} - y_i, x_{i+1} - x_i)
\end{align}
Délka strany $s_i$ je:
\begin{equation}
    s_i = \sqrt{(x_{i+1} - x_i)^2 + (y_{i+1} - y_i)^2}
\end{equation}
Vnitřní úhel $\omega_i$ je:
\begin{equation}
    \omega_i = |\sigma_i - \sigma_{i+1}|
\end{equation}
Násobek $\pi/2$ je:
\begin{equation}
    k_i = \frac{2\omega_i}{\pi}
\end{equation}
Orientovaný zbytek po dělení je:
\begin{equation}
    r_i = (k_i - \lfloor k_i \rfloor) \frac{\pi}{2}
\end{equation}
Pokud $\omega_i \mod \pi/2 > \pi/4$, pak:
\begin{equation}
    r_i = \frac{\pi}{2} - r_i
\end{equation}
Hlavní směr budovy $\sigma$ je:
\begin{equation}
    \sigma = \sigma_{1,2} + \frac{\sum_{i=1}^{n} r_i s_i}{\sum_{i=1}^{n} s_i}
\end{equation}
Dále je postup obdobný jako u předchozích metod.

\begin{algorithm}
    \setstretch{0.9}
    \caption{Metoda \texttt{Wall Average}}
    \begin{algorithmic}[1]
        \STATE \textbf{Vstup:} polygon $P$
        \STATE \textbf{Výstup:} obdélník představující generalizaci polygonu
        
        \STATE $n \gets \text{size}(P)$
        \STATE $\sigma \gets 0$
        \STATE $S_i\_sum \gets 0$
        
        \STATE $dx \gets x_1 - x_0$
        \STATE $dy \gets y_1 - y_0$
        \STATE $\sigma_{12} \gets \arctan2(dy, dx)$
        
        \FOR{$i \gets 0$ \TO $n-1$}
            \STATE $dx1 \gets x_{i-1} - x_i$
            \STATE $dy1 \gets y_{i-1} - y_i$
            \STATE $\sigma_i \gets \arctan2(dy1, dx1)$
            
            \STATE $dx2 \gets x_{i+1} - x_i$
            \STATE $dy2 \gets y_{i+1} - y_i$
            \STATE $\sigma_{i+1} \gets \arctan2(dy2, dx2)$
            \STATE $S_i \gets \sqrt{dx2^2 + dy2^2}$
            
            \STATE $\omega_i \gets |\sigma_i - \sigma_{i+1}|$
            \STATE $k_i \gets \frac{2 \omega_i}{\pi}$
            \STATE $r_i \gets (k_i - \lfloor k_i \rfloor) \frac{\pi}{2}$
            
            \IF{$\omega_i \mod \pi/2 > \pi/4$}
                \STATE $r_i \gets \frac{\pi}{2} - r_i$
            \ENDIF
            
            \STATE $\sigma \gets \sigma + r_i S_i$
            \STATE $S_i\_sum \gets S_i\_sum + S_i$
        \ENDFOR
        
        \STATE $\sigma \gets \sigma_{12} + \frac{\sigma}{S_i\_sum}$
        \STATE $P_{\text{rot}} \gets \text{rotate}(P, -\sigma)$
        \STATE $\text{mmbox}, \text{area} \gets \text{minMaxBox}(P_{\text{rot}})$
        
        \STATE $\text{mmbox\_min\_res} \gets \text{resize}(P, \text{mmbox})$
        \STATE \textbf{return} \text{rotate}($\text{mmbox\_min\_res}, \sigma$)
    \end{algorithmic}
\end{algorithm}

\subsubsection{Weighted Bisector}
Metoda, která určuje natočení obdélníku na základě váženého průměru orientací hran polygonu. Váhy jsou přiřazeny jednotlivým hranám podle jejich délky. Velikost obdélníku je volena tak, aby měl obdélník obsah shodný s polygonem. Těžiště obdélníku je shodné s těžištěm polygonu.\\
Nejprve se najdou dvě nejdelší úhlopříčky polygonu \( P \):
\begin{equation}
    \text{max\_diag\_1} = 0, \quad \text{max\_diag\_2} = 0
\end{equation}
Pro každou dvojici vrcholů \( p_i \) a \( p_j \) se vypočítají rozdíly souřadnic:
\begin{equation}
    dx_i = x_j - x_i
\end{equation}
\begin{equation}
    dy_i = y_j - y_i
\end{equation}
Délka úhlopříčky se vypočítá jako:
\begin{equation}
    \text{len}_i = \sqrt{dx_i^2 + dy_i^2}
\end{equation}
Pokud je délka úhlopříčky větší než \( \text{max\_diag\_1} \), aktualizuje se \( \text{max\_diag\_2} \) a \( \text{max\_diag\_1} \) a směrové rozdíly \( dx \) a \( dy \):
\begin{equation}
    \text{if } \text{len}_i > \text{max\_diag\_1} \text{ then}
\end{equation}
\begin{equation}
    \quad \text{max\_diag\_2} = \text{max\_diag\_1}
\end{equation}
\begin{equation}
    \quad \text{max\_diag\_1} = \text{len}_i
\end{equation}
\begin{equation}
    \quad dx_2 = dx_1, \quad dy_2 = dy_1
\end{equation}
\begin{equation}
    \quad dx_1 = dx_i, \quad dy_1 = dy_i
\end{equation}
Pokud je délka úhlopříčky větší než \( \text{max\_diag\_2} \) a menší než \( \text{max\_diag\_1} \), aktualizuje se \( \text{max\_diag\_2} \) a směrové rozdíly \( dx \) a \( dy \):
\begin{equation}
    \text{else if } \text{len}_i > \text{max\_diag\_2} \text{ and } \text{len}_i < \text{max\_diag\_1} \text{ then}
\end{equation}
\begin{equation}
    \quad \text{max\_diag\_2} = \text{len}_i
\end{equation}
\begin{equation}
    \quad dx_2 = dx_i, \quad dy_2 = dy_i
\end{equation}
Směr nejdelších úhlopříček se vypočítá jako:
\begin{equation}
    \sigma_1 = \arctan2(dy_1, dx_1)
\end{equation}
\begin{equation}
    \sigma_2 = \arctan2(dy_2, dx_2)
\end{equation}
Vážený průměr směru úhlopříček se vypočítá jako:
\begin{equation}
    \sigma = \frac{\sigma_1 \cdot \text{max\_diag\_1} + \sigma_2 \cdot \text{max\_diag\_2}}{\text{max\_diag\_1} + \text{max\_diag\_2}}
\end{equation}
Dále je postup obdobný jako u předchozích metod.
\newpage
\begin{algorithm}
    \setstretch{0.9}
    \caption{Metoda \texttt{Weighted Bisector}}
    \begin{algorithmic}[1]
        \STATE \textbf{Vstup:} polygon $P$
        \STATE \textbf{Výstup:} obdélník představující generalizaci polygonu
        
        \STATE $\text{max\_diag\_1} \gets 0$, $\text{max\_diag\_2} \gets 0$
        \STATE $dx_1 \gets 0$, $dy_1 \gets 0$, $dx_2 \gets 0$, $dy_2 \gets 0$
        \STATE $n \gets \text{size}(P)$
        
        \FOR{$i \gets 0$ \TO $n-1$}
            \FOR{$j \gets 0$ \TO $n-1$}
                \STATE $dx_i \gets x_{i+j+2} - x_i$
                \STATE $dy_i \gets y_{i+j+2} - y_i$
                \STATE $\text{len}_i \gets \sqrt{dx_i^2 + dy_i^2}$
                
                \IF{$\text{len}_i > \text{max\_diag\_1}$}
                    \STATE $\text{max\_diag\_2} \gets \text{max\_diag\_1}$
                    \STATE $\text{max\_diag\_1} \gets \text{len}_i$
                    \STATE $dx_2 \gets dx_1$, $dy_2 \gets dy_1$
                    \STATE $dx_1 \gets dx_i$, $dy_1 \gets dy_i$
                \ELSIF{$\text{len}_i > \text{max\_diag\_2} \text{ and } \text{len}_i < \text{max\_diag\_1}$}
                    \STATE $\text{max\_diag\_2} \gets \text{len}_i$
                    \STATE $dx_2 \gets dx_i$, $dy_2 \gets dy_i$
                \ENDIF
            \ENDFOR
        \ENDFOR
        
        \STATE $\sigma_1 \gets \arctan2(dy_1, dx_1)$
        \STATE $\sigma_2 \gets \arctan2(dy_2, dx_2)$
        \STATE $\sigma \gets \frac{\sigma_1 \cdot \text{max\_diag\_1} + \sigma_2 \cdot \text{max\_diag\_2}}{\text{max\_diag\_1} + \text{max\_diag\_2}}$
        
        \STATE $P_{\text{rot}} \gets \text{rotate}(P, -\sigma)$
        \STATE $\text{mmbox}, \text{area} \gets \text{minMaxBox}(P_{\text{rot}})$
        
        \STATE $\text{mmbox\_min\_res} \gets \text{resize}(P, \text{mmbox})$
        \STATE \textbf{return} \text{rotate}($\text{mmbox\_min\_res}, \sigma$)
    \end{algorithmic}
\end{algorithm}

