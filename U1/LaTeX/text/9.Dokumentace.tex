\newpage
\section{Dokumentace}

\subsection{Použité knihovny}
Kód využívá následující knihovny:
\begin{itemize}
\item \texttt{\href{https://www.qt.io/}{Qt}} – Knihovna pro správu grafických objektů, včetně tříd jako \texttt{QPointF}, \texttt{QPolygonF}, \texttt{QPainterPath}.
\item \texttt{\href{http://shapelib.maptools.org/}{shapelib}} – Knihovna pro práci se soubory formátu Shapefile \cite{shapelib}. Upravená verze této knihovny do C++ byla použita z projektu \href{https://github.com/zhihao-liu/shapefile-viewer-qt/tree/master/shapelib}{shapefile-viewer-qt} od Zhihao Liu \cite{liu2025shapelib}.
\end{itemize}

\subsection{Třída Algorithms}
Třída poskytuje metody pro geometrické analýzy bodů a polygonů.

\textbf{Veřejné statické metody:}
\begin{itemize}
\item \texttt{short analyzeRayCrossing(const QPointF \&q, const QPolygonF \&pol)} – Analyzuje vztah bodu a polygonu pomocí algoritmu Ray Crossing.
\item \texttt{short analyzeWindingNumber(const QPointF \&q, const QPolygonF \&pol)} – Analyzuje vztah bodu a polygonu pomocí algoritmu Winding Number.
\item \texttt{bool minMaxBox(const QPointF \&q, const QPolygonF \&pol)} – Kontroluje, zda bod leží uvnitř minmaxboxu polygonu.
\end{itemize}

\textbf{Soukromé statické metody:}
\begin{itemize}
\item \texttt{double calculateDistance(const QPointF \&p1, const QPointF \&p2)} – Vypočítá euklidovskou vzdálenost mezi dvěma body (\texttt{p1} a \texttt{p2}).
\item \texttt{double calculateCosineValue(double l\_qi, double l\_qi1, double l\_ii1)} – Vypočítá kosinusový úhel mezi vektory.
\item \texttt{double calculateDeterminant(const QPointF \&p1, const QPointF \&p2, const QPointF \&q)} – Vypočítá determinant matice tvořené třemi body (\texttt{p1}, \texttt{p2}, \texttt{q}).
\item \texttt{short checkSingularities(const QPointF \&q, const QPolygonF \&pol, int i, int ii)} – Ověří výskyt singulárních bodů v polygonu.
\end{itemize}

\subsection{Třída Draw}
Třída pro vykreslování polygonů a interakci s uživatelem.

\textbf{Veřejné metody:}
\begin{itemize}
\item \texttt{void mousePressEvent(QMouseEvent *e)} – Zpracuje událost kliknutí myší.
\item \texttt{void paintEvent(QPaintEvent *event)} – Vykreslí polygony a body.
\item \texttt{void switch\_source()} – Přepíná typ vstupních dat.
\item \texttt{void loadPolygonFromFile(const QString \&fileName)} – Načte polygon z \texttt{*.TXT}.
\item \texttt{void loadPolygonFromShapefile(const QString \&fileName)} – Načte polygon z \texttt{*.SHP}.
\item \texttt{void clearPolygons()} – Vymaže všechny polygony.
\item \texttt{void addSelectedPolygon(const QPainterPath\& selection)} – Přidá vybraný polygon.
\item \texttt{void clearSelectedPolygons()} – Odstraní všechny vybrané polygony.
\end{itemize}

\textbf{Soukromé metody:}
\begin{itemize}
\item \texttt{void mousePressEventLeft(QMouseEvent *e)} – Zpracovává kliknutí levého tlačítka myši.
\item \texttt{void mousePressEventRight(QMouseEvent *e)} – Zpracovává kliknutí pravého tlačítka myši.
\item \texttt{void addPointToPath(QPointF p)} – Přidává bod \texttt{p} do aktuální cesty, která tvoří polygon.
\end{itemize}

\subsection{Třída MainForm}
Hlavní uživatelské rozhraní aplikace.

\textbf{Soukromé sloty:}
\begin{itemize}
\item \texttt{void on\_actionPoint\_Polygon\_triggered()} – Přepnutí vstupu bod/polygon
\item \texttt{void on\_actionRay\_Crossing\_triggered()} – Spustí analýzu metodou Ray Crossing.
\item \texttt{void on\_actionWinding\_Number\_triggered()} – Spustí analýzu metodou Winding Number.
\item \texttt{void on\_actionOpen\_triggered()} – Otevře nabídku pro výběr souboru.
\item \texttt{void on\_actionClear\_data\_triggered()} – Vymaže aktuální data.
\item \texttt{void on\_actionClear\_all\_triggered()} – Vymaže všechny objekty.
\item \texttt{void on\_actionExit\_triggered()} – Ukončí aplikaci.
\end{itemize}
