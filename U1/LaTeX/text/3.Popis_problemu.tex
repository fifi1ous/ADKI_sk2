\section{Popis problému}

Problém lokalizace bodu v polygonové mapě spočívá v určení toho, ve kterém polygonu se nachází daný bod. Tento úkol je klíčový v geografických informačních systémech (GIS) a je známý jako \textit{Point Location Problem}\cite{Bayer2008}.

\subsection{Formulace problému}

\paragraph{Dáno:}
\begin{itemize}
    \item Množina \( n \) bodů \(\{p_i\}\), které tvoří vrcholy \( m \) polygonů \(\{P_j\}\).
    \item Bod \( q \), jehož polohu vůči polygonům chceme určit.
\end{itemize}

\paragraph{Určováno:}
\begin{itemize}
    \item Polygon \( P \), který obsahuje bod \( q \), nebo informace, že bod \( q \) neleží v žádném polygonu.
\end{itemize}

\subsection{Techniky řešení problému}

\begin{itemize}
    \item \textbf{Half-plane test} - Spočívá v porovnání polohy bodu vůči každé hraně polygonu. Nelze jej využít pro nekonvexní polygony, kde mohou existovat hrany obklopující bod z různých stran. V praxi se tento problém řeší rozdělením nekonvexního polygonu na několik konvexních částí, na které lze test aplikovat jednotlivě.
    
    \item \textbf{Ray Crossing Algorithm} – Sleduje počet průsečíků polopřímky vycházející z bodu \( q \) s hranami polygonu. Pokud je počet průsečíků lichý, bod leží uvnitř polygonu; pokud je sudý, bod se nachází vně.
    
    \item \textbf{Winding Number Algorithm} – Vyhodnocuje součet úhlových změn mezi hranami polygonu a bodem. Pokud je výsledný součet nenulový, bod leží uvnitř polygonu.
\end{itemize}

